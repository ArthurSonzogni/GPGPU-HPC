%------------------------------------------------------------------------------%
%                                                                              %
%                                                                              %
%   EnsiRapport Template                                                       %
%                                                                              %
%   Version : 1.0                                                              %
%                                                                              %
%   Auteur : Arthur Sonzogni                                                   %
%                                                                              %
%------------------------------------------------------------------------------%



\documentclass[liens,entete-ensimag,margeCorrection]{ensirapport}
% options possibles:
% -  ('10pt', '11pt' and '12pt') taille de police.
% -  ('a4paper', 'letterpaper', 'a5paper', 'legalpaper', 'executivepaper' and 'landscape') taille de papier.
% -  ('sans' and 'roman') famille de police.
% -- ('liens') ajoute les liens dans le sommaire.
% -- ('entete,entete-ensimag') ajoute des belles entetes avec logo ensimag ou pas.
% -- ('margeCorrection') diminue les enormes marges de latex.
% -- ('minted') inclus minted pour colorer les codes sources, ne fonctionne pas à l'ensimag.
% -  ('onecolumn','twocolumn') une ou deux colonnes
% -  ('fleqn','leqno') formules mathématique alignées a gauche ou a droite.
% -  ('notitlepage','titlepage') sans ou avec page de garde pour le titre

%\usepackage[draft]{graphicx}
\usepackage[nottoc, notlof, notlot]{tocbibind}
\setlength{\parindent}{0cm} % Défini la largeur de l'alinéa de 1cm.



\begin{document}


\title{Mon titre}
\author{Arthur Sonzogni}
\date{\today}

\renewcommand{\labelitemi}{\textbullet}

\large
\thispagestyle{plain}

\begin{center}


Institut National Polytechnique de Grenoble

École Nationale Supérieure d'Informatique et de Mathématiques Appliquées de
Grenoble

\vspace{0.4cm}


{\huge \bfseries HPC \& GPGPU}

\vspace{0.5cm}
{\large \bfseries Rapport des travaux}


\vspace{1.5cm}

\hrule width \textwidth height 2pt
\vspace{0.4cm}
{\Huge \bfseries HPC \& GPGPU.}
\vspace{0.4cm}
\hrule width \textwidth height 2pt

\vspace{2cm}

\end{center}
\begin{minipage}{0.5\textwidth}
    
{\it Auteurs:}

Arthur SONZOGNI
Thomas Coeffic

3A Grenoble INP - Ensimag

Filière MMIS (Modélisation Mathématique, Images, Simulation)

\end{minipage}


\vspace{2cm}

\begin{center}
Grenoble, le \today
\end{center}

\newpage
\normalsize

%\maketitle
\tableofcontents

\section{Aide mémoire}

\paragraph{Séquential}

\paragraph{OpenMP\_Simple}
\paragraph{OpenMP\_GRID ???}

\paragraph{GPGPU\_simple} partitionement par agents

\paragraph{HPC\_MPI} partitionement par agents
\paragraph{HPC\_MPI\_GRID} partitionement par espace



\section{Présentation des différents algorithmes}
\subsection{Algorithme séquentiel}
\subsection{Parallèlisation par agents (OpenMP)}
\subsection{Parallèlisation par agents (MPI)}
\subsection{Parallèlisation par agents (CUDA)}
\subsection{Parallèlisation spatiale (Grille régulière) (MPI)}
Pour cet algorithme, on subdivise l'espace en une grille régulière de $(2^3{^n})$ blocks.
\subsection{Parallèlisation spatiale (Grille régulière) (CUDA)}

\end{document}
